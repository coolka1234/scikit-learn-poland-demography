\documentclass[11pt]{article}

\usepackage{sectsty}
\usepackage{graphicx}
\usepackage[T1]{fontenc}
\usepackage{multicol}

% Margins
\topmargin=-0.5in
\evensidemargin=0in
\oddsidemargin=0in
\textwidth=6.5in
\textheight=9.0in
\headsep=0.25in

\title{ Analiza możliwości wykorzystania modelów regresji do przewidywania liczby ludności w Polsce}
\author{ Krzysztof Kulka
        \\ 272667@student.pwr.edu.pl \\ MSiD Lab Wtorek 9.15 NP }
\date{\today}

\renewcommand{\contentsname}{Spis treści}

\begin{document}
\maketitle	
\pagebreak

% Optional TOC
\tableofcontents
 \pagebreak

%--Paper--

\section{Wstęp}
Problemem projektu jest analiza możliwości modelów regresji liniowej do przewidywania liczby ludności w Polsce. W tym celu wykorzystane zostaną dane historyczne dotyczące demografi, ludności, migracji oraz innych czynników wpływających na liczebność populacji.
Przedstawiona analiza ma na celu rozstrzygnięcie czy model regresji liniowej jest odpowiedni do przewidywania liczby ludności w Polsce, jakie modele jeśli tak, sprawdzają się do tego najlepiej oraz jakie czynniki mają największy wpływ na liczebność populacji.
Analizie zostaną poddane następujące czynniki:
\begin{itemize}
\item Oczekiwana długość życia
\item Imigracja do kraju
\item Wskaźnik dzietności
\item Wskaźnik urbanizacji
\item Historyczna populacja
\item Wskaźnik zmiany populacji na przestrzeni ostatnich 5 lat
\end{itemize}
Zbadane zaś zostaną następujące modele regresji:
\begin{itemize}
\item Regresja liniowa
\item Regresja typu Ridge
\item Regresja Decisions Trees
\item Regresja Random Forest
\item Regresja Lasso
\end{itemize}
\section{Zbiór danych i jego analiza}
Lorem Ipsum \\

%--/Paper--

\end{document}