\documentclass[11pt]{article}

\usepackage{sectsty}
\usepackage{graphicx}
\usepackage[T1]{fontenc}
\usepackage{multicol}
\usepackage{hyperref}

% Margins
\topmargin=-0.5in
\evensidemargin=0in
\oddsidemargin=0in
\textwidth=6.5in
\textheight=9.0in
\headsep=0.25in

\title{ Analiza możliwości wykorzystania modelów regresji do przewidywania liczby ludności w Polsce}
\author{ Krzysztof Kulka
        \\ 272667@student.pwr.edu.pl \\ MSiD Lab Wtorek 9.15 NP }
\date{\today}

\renewcommand{\contentsname}{Spis treści}

\begin{document}
\maketitle	
\pagebreak

% Optional TOC
\tableofcontents
 \pagebreak

%--Paper--

\section{Wstęp}
Problemem projektu jest analiza możliwości modelów regresji liniowej do przewidywania liczby ludności w Polsce. W tym celu wykorzystane zostaną dane historyczne dotyczące demografi, oraz innych czynników wpływających na liczebność populacji.
Przedstawiona analiza ma na celu rozstrzygnięcie czy model regresji liniowej jest odpowiedni do przewidywania liczby ludności w Polsce, oraz jakie modele sprawdzają się do tego najlepiej.
Analizie zostaną poddane następujące czynniki:
\begin{itemize}
\item Historyczna liczba ludności
\item Imigracja do kraju
\item Wskaźnik dzietności
\item Oczekiwana długość życia
\item Urbanizacja
\item Wskaźnik zmiany populacji na przestrzeni ostatnich 5 lat
\end{itemize}
Zbadane zaś zostaną następujące modele regresji:
\begin{itemize}
\item Regresja liniowa
\item Regresja typu Ridge
\item Regresja Decisions Trees
\item Regresja Random Forest
\item Regresja Lasso
\end{itemize}
\section{Zbiór danych i jego analiza}
\subsection*{Opis zbioru danych}
Zbiór danych zawiera informacje na temat liczby ludności, imigracji do kraju, wskazniku dzietnosci, oczekiwanej długości życia w momencie urodzenia na przestrzeni lat 1960-2023.
Dane zostały pobrane z serwisu internetowego World Bank\cite{wbd}. 
Dodatkowo informacje na temat urbanizacji
zostały pobrane z serwisu internetowego Zintegrowana Platforma Edukacyjna Ministerstwa Edukacji Narodowej\cite{zpe}, a wskaźnik zmiany populacji na przestrzeni ostatnich 5 lat został obliczony na podstawie danych historycznych.

\bibliographystyle{plain}
\bibliography{bibliografia}
\end{document}
