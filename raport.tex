\documentclass[11pt]{article}

\usepackage{sectsty}
\usepackage{graphicx}
\usepackage[T1]{fontenc}
\usepackage{multicol}
\usepackage{hyperref}
\usepackage{float}
% Margins
\topmargin=-0.5in
\evensidemargin=0in
\oddsidemargin=0in
\textwidth=6.5in
\textheight=9.0in
\headsep=0.25in

\title{ Analiza modelów regresji do przewidywania liczby ludności w Polsce}
\author{ Krzysztof Kulka
        \\ 272667@student.pwr.edu.pl \\ MSiD Lab Wtorek 9.15 NP }
\date{\today}

\renewcommand{\contentsname}{Spis treści}

\begin{document}
\maketitle	
\pagebreak

% Optional TOC
\tableofcontents
 \pagebreak

%--Paper--

\section{Wstęp}
Problemem projektu jest analiza możliwości modelów regresji liniowej do przewidywania liczby ludności w Polsce. W tym celu wykorzystane zostaną dane historyczne dotyczące demografi, oraz innych czynników wpływających na liczebność populacji.
Przedstawiona analiza ma na celu rozstrzygnięcie czy model regresji liniowej jest odpowiedni do przewidywania liczby ludności w Polsce, oraz jakie modele sprawdzają się do tego najlepiej.
Analizie zostaną poddane następujące czynniki:
\begin{itemize}
\item Historyczna liczba ludności
\item Imigracja do kraju
\item Wskaźnik dzietności
\item Oczekiwana długość życia
\item Urbanizacja
\item Wskaźnik zmiany populacji na przestrzeni ostatnich 5 lat
\end{itemize}
Zbadane zaś zostaną następujące modele regresji:
\begin{itemize}
\item Regresja liniowa
\item Regresja typu Ridge
\item Regresja Decisions Trees
\item Regresja Random Forest
\item Regresja Lasso
\end{itemize}
\section{Zbiór danych i jego analiza}
\subsection*{Opis zbioru danych}
Zbiór danych zawiera informacje na temat historycznej liczby ludności, imigracji do kraju, wskazniku dzietnosci, oczekiwanej długości życia w momencie urodzenia na przestrzeni lat 1960-2023.
Dane zostały pobrane z serwisu internetowego World Bank\cite{wbd} Dane dotyczą około 260 krajów. 
Dodatkowo informacje na temat urbanizacji
zostały pobrane z serwisu internetowego Zintegrowana Platforma Edukacyjna Ministerstwa Edukacji Narodowej\cite{zpe}, a wskaźnik zmiany populacji na przestrzeni ostatnich 5 lat został obliczony na podstawie danych historycznych.
\subsection*{Analiza danych}
Do analizy eksploracyjnej danych wykorzystano bibliotekę pandas-profiling\cite{pp}
\subsection*{Populacja w Polsce}
\begin{figure}[H]
        \centering
        \includegraphics[width=0.8\textwidth]{polish_population_over_the_years.png}
        \caption{Wizualizacja liczby ludności w Polsce na przestrzeni lat}
\end{figure}
\begin{figure}[H]
        \centering
        \includegraphics[width=0.8\textwidth]{images/histogram_populacja.png}
        \caption{Histogram liczby ludności w Polsce}
\end{figure}
\begin{table}[H]
        \centering
        \begin{tabular}{|l|l|l|}
        \hline
        Minimum & Maximum & Mediana \\ \hline
        29637450 & 38663481 & 37899070 \\ \hline
        \end{tabular}
        \caption{Statystyki liczby ludności w Polsce}
        \end{table}
\subsection*{Imigracja do Polski}
\begin{figure}[H]
        \centering
        \includegraphics[width=0.8\textwidth]{polish_int_migrant_stock_over_the_years.png}
        \caption{Wizualizacja imigracji do Polski na przestrzeni lat}
\end{figure}
\begin{figure}[H]
        \centering
        \includegraphics[width=0.8\textwidth]{images/histogram_imigracja.png}
        \caption{Histogram imigracji do Polski na przestrzeni lat}
\end{figure}
\begin{table}[H]
        \centering
        \begin{tabular}{|l|l|l|}
        \hline
        Minimum & Maximum & Mediana \\ \hline
        0 & 0 & 0 \\ \hline
        \end{tabular}
        \caption{Statystyki imigracji do Polski}
        \end{table}
\subsection*{Współczynnik dzietności}
\begin{figure}[H]
        \centering
        \includegraphics[width=0.8\textwidth]{polish_fertility_rate.png}
        \caption{Wizualizacja współczynnika dzietności Polski na przestrzeni lat}
\end{figure}
\begin{figure}[H]
        \centering
        \includegraphics[width=0.8\textwidth]{images/histogram_dzietnosc.png}
        \caption{Histogram współczynnika dzietności na przestrzeni lat}
\end{figure}
\begin{table}[H]
        \centering
        \begin{tabular}{|l|l|l|}
        \hline
        Minimum & Maximum & Mediana \\ \hline
        1.4 & 1.6 & 1.5 \\ \hline
        \end{tabular}
        \caption{Statystyki współczynnika dzietności}
        \end{table}
\subsection*{Oczekiwana długość życia}
\begin{figure}[H]
        \centering
        \includegraphics[width=0.8\textwidth]{polish_life_expentancy.png}
        \caption{Wizualizacja oczekiwanej długości życia na przestrzeni lat}
\end{figure}
\begin{figure}[H]
        \centering
        \includegraphics[width=0.8\textwidth]{images/histogram_dl_zycia.png}
        \caption{Histogram oczekiwanej długości życia na przestrzeni lat}
\end{figure}
\begin{table}[H]
        \centering
        \begin{tabular}{|l|l|l|}
        \hline
        Minimum & Maximum & Mediana \\ \hline
        70.0 & 80.0 & 75.0 \\ \hline
        \end{tabular}
        \caption{Statystyki oczekiwanej długości życia}
        \end{table}
\subsection*{Urbanizacja}
\begin{figure}[H]
        \centering
        \includegraphics[width=0.8\textwidth]{polish_urbanization.png}
        \caption{Wizualizacja urbanizacji na przestrzeni lat}
\end{figure}
\begin{figure}[H]
        \centering
        \includegraphics[width=0.8\textwidth]{images/histogram_urbanizacja.png}
        \caption{Wizualizacja urbanizacji na przestrzeni lat}
\end{figure}
\begin{table}[H]
        \centering
        \begin{tabular}{|l|l|l|}
        \hline
        Minimum & Maximum & Mediana \\ \hline
        0.0 & 0.0 & 0.0 \\ \hline
        \end{tabular}
        \caption{Statystyki urbanizacji}
        \end{table}
\subsection*{Wskaźnik zmiany populacji}
\begin{figure}[H]
        \centering
        \includegraphics[width=0.8\textwidth]{polish_pop_change_5_years.png}
        \caption{Wizualizacja wskaźnika zmiany populacji na przestrzeni lat}
\end{figure}
\begin{figure}[H]
        \centering
        \includegraphics[width=0.8\textwidth]{images/histogram_zmiana_populacji.png}
        \caption{Histogram zmiany polskiej populacji na przestrzeni ostatnich 5 lat}
\end{figure}
\begin{table}[H]
        \centering
        \begin{tabular}{|l|l|l|}
        \hline
        Minimum & Maximum & Mediana \\ \hline
        -0.0001 & 0.0001 & 0.0 \\ \hline
        \end{tabular}
        \caption{Statystyki wskaźnika zmiany populacji}
        \end{table}
\subsection*{Korelacja}
\begin{figure}[H]
        \centering
        \includegraphics[width=0.8\textwidth]{images/matryca_korelacji.png}
        \caption{Macierz korelacji}
\end{figure}
        Z racji ręcznego dobioru danych i ich selekcji, analiza wykazała bardzo dużą korelację (oraz antykorelacje) pomiędzy wybranymi współczynnikami.
        Pojawia się bardzo zaskakująca, przecząca logice korelacja, pomiędzy współczynnikiem dzietności a populacją -0.763.
        Prawdopodobnie wynika ona z tego że przez większość badanego okresu, współczynnik był nadal na bardzo wysokim poziomie, więc mimo że nie malał, to populacja stale się zwiększała.
        Analogicznie zaskakuje negatywna korelacja migracji z populacją, co jest zaskakujące, ponieważ zgodnie z intuicją, imigracja powinna zwiększać populację.
        Prawdopodobnie wynika to z faktu, że dane dotyczące imigracji są niewielkie, co powoduje że słabo, choć wciąż, przekładają się na polską populację.
        Jeśli zaś chodzi o spodziewane korelacje, należy szczególnie zwrócić uwagę:
        \begin{itemize}
        \item Oczekiwana długość życia z populacją: 0.683
        \item Urbanizacja z populacją: 0.949
        \item Urbanizacja z oczekiwaną długością życia: 0.611
        \end{itemize}
        Korelacje te dobrze wróżą dla modeli regresji, ponieważ są one na tyle silne, że powinny pozwolić na skuteczne przewidywanie liczby ludności w Polsce.
\subsection*{Obróbka danych}
\subsubsection*{Pozyskanie danych}
Najważniejszym krokiem w obróbce danych było wyizolowanie danych dotyczących Polski, oraz usunięcie kolumn, które nie były istotne dla analizy.
Dodatkowo, z racji tego że dane dotyczące imigracji rejestrowane co pięć lat, skorzystano z interpolacji liniowej, aby uzupełnić brakujące dane.
Największe wyzwanie pojawiło się z danymi dotyczącymi urbanizacji. Jedyne z zaufanego oficjalnego źródła były w formie obrazka .jpg.
Aby więc uzyskać dane, i móc zamienić je w data frame, użyto następujących kroków:
\begin{itemize}
\item Scraping obrazka za pomocą skryptu pythonowego
\item Następnie przekazanie zescrapowanego obrazka do zewnętrznego programu WebPlotDigitizer\cite{wpd}
\item Finalnie, z racji niedoskonałości otrzymanego wyniku (m.in. potraktowanie lat jako liczb rzeczywistych, a nie całkowitych), dane zostały poprawione za pomocą kolejnego skryptu napisanego w pythonie. 
\end{itemize}
\subsubsection*{Eliminacja danych odstających}
Podczas przeglądania grafów reprezentujących zebrane dane, nie trudno było zauważyć że dane po 2019 roku są znacznie odstające od reszty.
Oczywistym jest że w 2020 roku, z racji pandemii, wiele wskaźników uległo zmianie, co sprawia że dane z tego roku są nieprzydatne do analizy.
Z tego powodu, dane z 2020 roku wzwyż zostały usunięte.
\bibliographystyle{plain}
\bibliography{bibliografia}
\end{document}
